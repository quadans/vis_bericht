\documentclass[portrait,a0,final]{a0poster}
%
\usepackage[english]{babel}
\usepackage{calc}
\usepackage{amsmath}
\usepackage{amsthm}
\usepackage{epsfig}
\usepackage{graphicx}
\usepackage{tikz}
\usetikzlibrary{calc}
\usetikzlibrary{shapes}
\usetikzlibrary{positioning}
\usepackage{booktabs}
\usepackage{setspace}
\usepackage{shadow}
\usepackage{multicol}
\usepackage{amssymb}
\usepackage{wasysym}                   %\permil
\usepackage[autolanguage]{numprint}    %\numprint[Nm]{129313} -> 129,313 Nm
\usepackage[T1]{fontenc}
\usepackage[utf8]{inputenc}
\usepackage{caption}
%\usepackage{color}
\usepackage{wrapfig}
\usepackage[notransparent]{svg}
\usepackage{amsmath} % you need amsmath as the demo includes a use of \eqref
\usepackage{stackengine}% for put note about sorce below image
\usepackage{tabularx}
\usepackage{subcaption}


%\usepackage{natbib}
\bibliographystyle{abbrv}

\xdefinecolor{unigruen}{RGB}{132,184,24}
\xdefinecolor{mainCol}{rgb}{0.6,0.8,1}
\xdefinecolor{picCol}{rgb}{0.9,0.6,1}
\xdefinecolor{BoxCol}{rgb}{0.85,0.6,1}
\xdefinecolor{TextCol}{rgb}{0,0,0}


% Weitere Schriftarten, wird hier aber nicht unbedingt benötigt
\usepackage{ae,mathpazo}
\renewcommand{\rmdefault}{phv}
\renewcommand{\sfdefault}{phv}
\renewcommand{\ttdefault}{aett}
\usepackage{sectsty}
\allsectionsfont{\usefont{OT1}{phv}{b}{n}\selectfont}

\newcommand*{\mysec}[1]{\medskip\stepcounter{section}%
\noindent\tikz{\node[fill=unigruen!99, rectangle, rounded corners, text width=\linewidth, text = white] {\Large\thesection~~~\textbf{#1}};}\par}

% Anpassung der Seitengröße
\setlength{\textwidth}{0.98\textwidth}

% Layout für den Spaltentext
\renewcommand{\thesection}{\arabic{section}.}
\setlength{\columnseprule}{2pt}
\setlength{\columnsep}{4cm}
\onehalfspacing
%\linespread{1.5}
\setlength{\parskip}{3pt plus 2pt minus 2pt}
\newcommand*{\HRuleFill}[1][.1pt]{%
\leavevmode\leaders\hrule height#1\hfill\kern0pt\relax}

\begin{document}
\fontfamily{phv}\selectfont
%%%%%%%%%%%%%%%%%%%%%%%%%%%%%%%%%%%%%%%%%%% Kopf %%%%%%%%%%%%%%%%%%%%%%%%%%%%%%%%
\begin{minipage}[c][3cm][c]{0.4\paperwidth}
     \includegraphics[height=3cm,scale=0.01]{tu_dortmund.jpg}\vfill\mbox{ }
\end{minipage} \hfill
\begin{minipage}[c][8cm][c]{0.5\paperwidth}
 \begin{flushright}
    {\huge Statistische Visualisierung / \\Visualisierung komplexer Datenstrukturen}\\[0.75ex]
    {\Large M. Sc. Julian Riehl, Prof. Dr. Katja Ickstadt} \\[0.75ex]
    {\Large Fakult\"at Statistik}\\[0.75ex]
 %   {\large{}}
 \end{flushright}
\end{minipage}
%%%%%%%%%%%%%%%%%%%%%%%%%%%%%%%%%%%%%%%%%% Titel%%%%%%%%%%%%%%%%%%%%%%%%%%%%%%%%%%%%
\bigskip\medskip
\vspace*{0.5cm}
\begin{minipage}{\textwidth}
 \begin{center}\veryHuge
  \textbf{\textsf{Datenauswertung von Neugeborenen und deren Mütter}}\\ \vspace{1cm}
\LARGE{Dennis Koleda, Patrick Wisniewski}
\vspace{-0.5cm}
 \end{center}
\end{minipage}
\smallskip
%\vspace*{-1em}

\rule{\linewidth}{3pt}\\ \vspace*{3cm}
%%%%%%%%%%%%%%%%%%%%%%%%%%%%%%%%%%%%%%%%% Mittelteil %%%%%%%%%%%%%%%%%%%%%%%%%%%%
\vspace*{-3em}
\begin{multicols}{2}

\mysec{Allgemeine Korrelationen}
\includegraphics[scale = 1]{../plots/6. Korrelationen.pdf}


(fig.1) Die Merkmale Gewicht, Kopfumfang und Größe des Kindes, sowie die Schwangerschaftsdauer hängen stark miteinander zusammen. Der Korrelationskoeffizient liegt zwischen $0,41$ und $0,76$. Der Zusammenhang ist beim Gewicht besonder stark. 

\mysec{Muttergröße mit Geburtsgewicht}

\includegraphics[scale = 0.5]{../plots/7. KorrGrMGB.pdf}

(fig. 2) Das Geburtsgewicht des Kindes korreliert schwach mit der Körpergröße der Mutter. Der Korrelationskoeffizient beträgt $0,13$.
\\
Größere Mütter bekommen tatsächlich im Durchschnitt schwerere Kinder!
%\vspace*{25cm}
\\

%%%%%%%%%%%%%%%%%%%%%%%%%%%%%%%%%%%%%%%%%%%%%%%%%%%%%%%%%%%%%%%%%%%%%%%%%%%%%%%%%
\mysec{Vitale Funktionen der Neugeborenen nach der Geburt}

\includegraphics[scale = 1]{../plots/1. Gesundheitsverlauf.pdf}

(fig. 3) Hier werden den Neugeborenen anhand verschiedener Vitalwerte die Kategorien 1 bis 10 zugeordnet.


\mysec{Vitale Funktionen nach Alter}

\includegraphics[scale = 0.8]{../plots/2. GesMutAPG.pdf}

(fig. 4) Hier wurden alle 3 APGAR Werte (die Werte welche die vitalen Funktionen kennziffern) nach jedem
Alter sortiert und gruppiert.

%%%%%%%%%%%%%%%%%%%%%%%%%%%%%%%%%%%%%%%%%%%%%%%%%%%%%%%%%%%%%%%%%%%%%%%%%%%%%%%%%
\mysec{Rauchen und Schwangerschaft?}

Ein doch sehr heikles Thema. Wie sehr beeinflusst Rauchen die Schwangerschaft? Dazu waren im Datensatz von 1000 Daten, 137 Raucherinnen. 

\includegraphics[scale = 1]{../plots/3. AnzahlRaucher.pdf}

(fig. 5) Der Mittelwert der beiden Gruppen unterscheided sich um 2 1/2 Jahre. (27.17 vs 29.77). Genauere Auswertung dann in der Präsentation!

\mysec{Rauchen und Einfluss auf das Kind}

Welchen Einfluss hat Rauchen auf's Kind? Dazu haben wir das (fig. 6) Geburtsgewicht untersucht sowie die vitalen Funktionen (fig. 7) nach der Geburt.

\includegraphics[scale = 0.9]{../plots/4. GbgewRaucher.pdf}

(fig. 6) Nichtraucher bekommen im Durchschnitt schwerere Kinder. Auch Ausreißer sind häufiger vertreten bei Nichtrauchern, allerdings ist die Datenlage bei Rauchern auch kleiner. 

\includegraphics[scale = 1]{../plots/5. APGraucher2.pdf}

(fig. 7) Dazu einmal die durchschnittlichen Vitalwerte.
%%%%%%%%%%%%%%%%%%%%%%%%%%%%%%%%%%%%%%%%%%%%%%%%%%%%%%%%%%%%%%%%%%%%%%%%%%%%%%%%%
%%%%%%%%%%%%%%%%%%%%%%%%%%%%%%%%%%%%%%%%%%%%%%%%%%%%%%%%%%%%%%%%%%%%%%%%%%%%%%%%%
%%%%%%%%%%%%%%%%%%%%%%%%%%%%%%%%%%%%%%%%%%%%%%%%%%%%%%%%%%%%%%%%%%%%%%%%%%%%%%%%%
\singlespacing
\renewcommand{\refname}{Quellen}
  % {\small
  %\bibliography{Literatur}}
  % ODER
 \begin{thebibliography}{99}
  \bibitem[1] DDatensatz: simulierter Datensatz birth.Rdata aus dem Modul (nachempfunden nach einem echten Geburtsdatensatz der Jahre 2003 - 2007 aus NRW)
  \bibitem[2] SSoftware und Pakete verwendet : LaTeX, RStudio, ggplot
 \end{thebibliography}
\end{multicols}
\end{document}

